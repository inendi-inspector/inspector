\chapter{Formats}

Formats allow Picviz to know what to do with files read. From title to the variable that is recommended to plot data on each axis. Every input needs a format.

\newpage

The easiest way to understand a format is to have a look at the ones used by pcre normalization plugins. They are installed by default in the directory \textit{normalize-helpers/pcre}. 

\begin{lstlisting}
# Rule created to parse iptable logs
revision = 1

pcre = "(\w+\s+\d+\s+\d+:\d+:\d+).*SRC=(\d+\.\d+\.\d+\.\d+).*DST=(\d+\.\d+\.\d+\.\d+).*LEN=(\d+).*TTL=(\d+).*ID=(\d+).*PROTO=(\S+).*SPT=(\d+).*DPT=(\d+).*"

# Feb 9 22:46:32
time-format[1] = "%b %d %H:%M:%S"

key-axes = "2,3,9"

axes {
   time 24h default "Time"
   ipv4 default default "Source"
   ipv4 default default "Destination"
   integer default minmax "Length"
   integer default minmax "TTL"
   integer default default "ID"
   enum default default "Protocol"
   integer default port "Source Port"
   integer default port "Destination Port"
}
\end{lstlisting}

\section{Format for CSV files}
CSV files can be given a format which can be one of those two files:
\begin{itemize}
\item \textbf{file.csv.format}: Add the \textbf{.format} extension to a given CSV file to use it when opening the file.
\item \textbf{picviz.format}: If it is in a directory, use it to open any CSV file contained in it. Useful when you have several CSV files to open.
\end{itemize}
